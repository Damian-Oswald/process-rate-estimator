% Options for packages loaded elsewhere
\PassOptionsToPackage{unicode}{hyperref}
\PassOptionsToPackage{hyphens}{url}
\PassOptionsToPackage{dvipsnames,svgnames,x11names}{xcolor}
%
\documentclass[
]{article}

\usepackage{amsmath,amssymb}
\usepackage{setspace}
\usepackage{iftex}
\ifPDFTeX
  \usepackage[T1]{fontenc}
  \usepackage[utf8]{inputenc}
  \usepackage{textcomp} % provide euro and other symbols
\else % if luatex or xetex
  \usepackage{unicode-math}
  \defaultfontfeatures{Scale=MatchLowercase}
  \defaultfontfeatures[\rmfamily]{Ligatures=TeX,Scale=1}
\fi
\usepackage{lmodern}
\ifPDFTeX\else  
    % xetex/luatex font selection
\fi
% Use upquote if available, for straight quotes in verbatim environments
\IfFileExists{upquote.sty}{\usepackage{upquote}}{}
\IfFileExists{microtype.sty}{% use microtype if available
  \usepackage[]{microtype}
  \UseMicrotypeSet[protrusion]{basicmath} % disable protrusion for tt fonts
}{}
\makeatletter
\@ifundefined{KOMAClassName}{% if non-KOMA class
  \IfFileExists{parskip.sty}{%
    \usepackage{parskip}
  }{% else
    \setlength{\parindent}{0pt}
    \setlength{\parskip}{6pt plus 2pt minus 1pt}}
}{% if KOMA class
  \KOMAoptions{parskip=half}}
\makeatother
\usepackage{xcolor}
\usepackage[margin=2cm]{geometry}
\setlength{\emergencystretch}{3em} % prevent overfull lines
\setcounter{secnumdepth}{5}
% Make \paragraph and \subparagraph free-standing
\ifx\paragraph\undefined\else
  \let\oldparagraph\paragraph
  \renewcommand{\paragraph}[1]{\oldparagraph{#1}\mbox{}}
\fi
\ifx\subparagraph\undefined\else
  \let\oldsubparagraph\subparagraph
  \renewcommand{\subparagraph}[1]{\oldsubparagraph{#1}\mbox{}}
\fi


\providecommand{\tightlist}{%
  \setlength{\itemsep}{0pt}\setlength{\parskip}{0pt}}\usepackage{longtable,booktabs,array}
\usepackage{calc} % for calculating minipage widths
% Correct order of tables after \paragraph or \subparagraph
\usepackage{etoolbox}
\makeatletter
\patchcmd\longtable{\par}{\if@noskipsec\mbox{}\fi\par}{}{}
\makeatother
% Allow footnotes in longtable head/foot
\IfFileExists{footnotehyper.sty}{\usepackage{footnotehyper}}{\usepackage{footnote}}
\makesavenoteenv{longtable}
\usepackage{graphicx}
\makeatletter
\def\maxwidth{\ifdim\Gin@nat@width>\linewidth\linewidth\else\Gin@nat@width\fi}
\def\maxheight{\ifdim\Gin@nat@height>\textheight\textheight\else\Gin@nat@height\fi}
\makeatother
% Scale images if necessary, so that they will not overflow the page
% margins by default, and it is still possible to overwrite the defaults
% using explicit options in \includegraphics[width, height, ...]{}
\setkeys{Gin}{width=\maxwidth,height=\maxheight,keepaspectratio}
% Set default figure placement to htbp
\makeatletter
\def\fps@figure{htbp}
\makeatother
\newlength{\cslhangindent}
\setlength{\cslhangindent}{1.5em}
\newlength{\csllabelwidth}
\setlength{\csllabelwidth}{3em}
\newlength{\cslentryspacingunit} % times entry-spacing
\setlength{\cslentryspacingunit}{\parskip}
\newenvironment{CSLReferences}[2] % #1 hanging-ident, #2 entry spacing
 {% don't indent paragraphs
  \setlength{\parindent}{0pt}
  % turn on hanging indent if param 1 is 1
  \ifodd #1
  \let\oldpar\par
  \def\par{\hangindent=\cslhangindent\oldpar}
  \fi
  % set entry spacing
  \setlength{\parskip}{#2\cslentryspacingunit}
 }%
 {}
\usepackage{calc}
\newcommand{\CSLBlock}[1]{#1\hfill\break}
\newcommand{\CSLLeftMargin}[1]{\parbox[t]{\csllabelwidth}{#1}}
\newcommand{\CSLRightInline}[1]{\parbox[t]{\linewidth - \csllabelwidth}{#1}\break}
\newcommand{\CSLIndent}[1]{\hspace{\cslhangindent}#1}

\usepackage[version=4]{mhchem}
\makeatletter
\makeatother
\makeatletter
\makeatother
\makeatletter
\@ifpackageloaded{caption}{}{\usepackage{caption}}
\AtBeginDocument{%
\ifdefined\contentsname
  \renewcommand*\contentsname{Table of contents}
\else
  \newcommand\contentsname{Table of contents}
\fi
\ifdefined\listfigurename
  \renewcommand*\listfigurename{List of Figures}
\else
  \newcommand\listfigurename{List of Figures}
\fi
\ifdefined\listtablename
  \renewcommand*\listtablename{List of Tables}
\else
  \newcommand\listtablename{List of Tables}
\fi
\ifdefined\figurename
  \renewcommand*\figurename{Figure}
\else
  \newcommand\figurename{Figure}
\fi
\ifdefined\tablename
  \renewcommand*\tablename{Table}
\else
  \newcommand\tablename{Table}
\fi
}
\@ifpackageloaded{float}{}{\usepackage{float}}
\floatstyle{ruled}
\@ifundefined{c@chapter}{\newfloat{codelisting}{h}{lop}}{\newfloat{codelisting}{h}{lop}[chapter]}
\floatname{codelisting}{Listing}
\newcommand*\listoflistings{\listof{codelisting}{List of Listings}}
\makeatother
\makeatletter
\@ifpackageloaded{caption}{}{\usepackage{caption}}
\@ifpackageloaded{subcaption}{}{\usepackage{subcaption}}
\makeatother
\makeatletter
\@ifpackageloaded{tcolorbox}{}{\usepackage[skins,breakable]{tcolorbox}}
\makeatother
\makeatletter
\@ifundefined{shadecolor}{\definecolor{shadecolor}{rgb}{.97, .97, .97}}
\makeatother
\makeatletter
\makeatother
\makeatletter
\makeatother
\ifLuaTeX
  \usepackage{selnolig}  % disable illegal ligatures
\fi
\IfFileExists{bookmark.sty}{\usepackage{bookmark}}{\usepackage{hyperref}}
\IfFileExists{xurl.sty}{\usepackage{xurl}}{} % add URL line breaks if available
\urlstyle{same} % disable monospaced font for URLs
\hypersetup{
  pdftitle={Process Rate Estimator},
  pdfauthor={Damian Oswald},
  colorlinks=true,
  linkcolor={blue},
  filecolor={Maroon},
  citecolor={Blue},
  urlcolor={Blue},
  pdfcreator={LaTeX via pandoc}}

\title{Process Rate Estimator}
\usepackage{etoolbox}
\makeatletter
\providecommand{\subtitle}[1]{% add subtitle to \maketitle
  \apptocmd{\@title}{\par {\large #1 \par}}{}{}
}
\makeatother
\subtitle{A modeling side-hustle for the ETH group sustainable
agroecosystems}
\author{Damian Oswald}
\date{September 23, 2023}

\begin{document}
\maketitle
\ifdefined\Shaded\renewenvironment{Shaded}{\begin{tcolorbox}[breakable, enhanced, sharp corners, interior hidden, boxrule=0pt, borderline west={3pt}{0pt}{shadecolor}, frame hidden]}{\end{tcolorbox}}\fi

\setstretch{1.25}
\[
\def\Ds{\text{D}_{\text{s}}}
\]

\hypertarget{introduction}{%
\section{Introduction}\label{introduction}}

Denitrification is the natural process by which nitrate
(NO\textsubscript{3}\textsuperscript{-}) in the soil are converted by
bacteria into nitrous oxide (N\textsubscript{2}O) or pure nitrigen
(N\textsubscript{2}). The latter is called \emph{total denitrification}
--- the full process described in Equation~\ref{eq-denitrification}
takes place.

\begin{equation}\protect\hypertarget{eq-denitrification}{}{
\ce{NO3^- ->[\text{Nitrate}][\text{reductase}] NO2^- ->[\text{Nitrite}][\text{reductase}] NO ->[\text{Nitrite oxide}][\text{reductase}] N2O^- ->[\text{Nitrous oxide}][\text{reductase}] N2}
}\label{eq-denitrification}\end{equation}

Denitrification occurs in conditions where oxygen is limited, such as
waterlogged soils. It is part of the nitrogen cycle, where nitrogen is
circulated between the atmosphere, organisms and the earth.

\hypertarget{formal-model-description}{%
\section{Formal model description}\label{formal-model-description}}

\hypertarget{model-parameters}{%
\subsection{Model parameters}\label{model-parameters}}

\hypertarget{tbl-parameters}{}
\begin{longtable}[]{@{}
  >{\raggedright\arraybackslash}p{(\columnwidth - 8\tabcolsep) * \real{0.1000}}
  >{\raggedright\arraybackslash}p{(\columnwidth - 8\tabcolsep) * \real{0.1500}}
  >{\raggedright\arraybackslash}p{(\columnwidth - 8\tabcolsep) * \real{0.5500}}
  >{\raggedright\arraybackslash}p{(\columnwidth - 8\tabcolsep) * \real{0.1500}}
  >{\raggedright\arraybackslash}p{(\columnwidth - 8\tabcolsep) * \real{0.0500}}@{}}
\caption{\label{tbl-parameters}Overview of the parameters used in the
model.}\tabularnewline
\toprule\noalign{}
\begin{minipage}[b]{\linewidth}\raggedright
Symbol
\end{minipage} & \begin{minipage}[b]{\linewidth}\raggedright
Code
\end{minipage} & \begin{minipage}[b]{\linewidth}\raggedright
Name
\end{minipage} & \begin{minipage}[b]{\linewidth}\raggedright
Value
\end{minipage} & \begin{minipage}[b]{\linewidth}\raggedright
Unit
\end{minipage} \\
\midrule\noalign{}
\endfirsthead
\toprule\noalign{}
\begin{minipage}[b]{\linewidth}\raggedright
Symbol
\end{minipage} & \begin{minipage}[b]{\linewidth}\raggedright
Code
\end{minipage} & \begin{minipage}[b]{\linewidth}\raggedright
Name
\end{minipage} & \begin{minipage}[b]{\linewidth}\raggedright
Value
\end{minipage} & \begin{minipage}[b]{\linewidth}\raggedright
Unit
\end{minipage} \\
\midrule\noalign{}
\endhead
\bottomrule\noalign{}
\endlastfoot
\(BD\) & \texttt{BD} & Bulk density (mass of the many particles of the
material divided by the bulk volume) & \(1.686\) & g
cm\textsuperscript{-3} \\
\(\theta_w\) & \texttt{theta\_w} & Soil volumetric water content & & \\
\(\theta_a\) & \texttt{theta\_a} & Air-filled porosity & & \\
\(\theta_t\) & \texttt{theta\_t} & Total soil porosity &
\(1-\frac{BD}{2.65}\) & \\
\(\text T\) & \texttt{temperature} & Soil temperature & \(298\) & K \\
\(\Ds\) & \texttt{D\_s} & Gas diffusion coefficient &
Equation~\ref{eq-Ds} & m\textsuperscript{2}s\textsuperscript{-1} \\
\(D_{\text{fw}}\) & \texttt{D\_fw} & Diffusivity of N\textsubscript{2}O
in water & Equation~\ref{eq-Dfw} & \\
\(D_{\text{fa}}\) & \texttt{D\_fa} & Diffusivity of N\textsubscript{2}O
in air & Equation~\ref{eq-Dfa} & \\
\(D_{\text{fa,NTP}}\) & & Free air diffusion coefficient under standard
conditions & Equation~\ref{eq-Dfa} & \\
\(n\) & \texttt{n} & Empirical
parameter\textsuperscript{\protect\hyperlink{ref-massman1998review}{1}}
& 1.81 & \\
\(H\) & \texttt{H} & Dimensionless Henry's solubility constant &
Equation~\ref{eq-H} & \\
\(\rho\) & \texttt{rho} & Gas density of N\textsubscript{2}O &
\(1.26 \times 10^6\) & \\
\end{longtable}

The diffusion fluxes between soil increments are described by Frick's
law (Equation~\ref{eq-frick}).

\begin{equation}\protect\hypertarget{eq-frick}{}{F_{\text{calc}} = \frac{dC}{dZ} D_{\text s} \rho}\label{eq-frick}\end{equation}

Here, \(D_s\) is the gas diffusion coefficient, \(\rho\) is the gas
density of N\textsubscript{2}O, and \(\frac{dC}{dZ}\) is the
N\textsubscript{2}O concentration gradient from lower to upper depth.
The fluxes are calculated based on N\textsubscript{2}O concentration
gradients between 105-135 cm, 75-105 cm, 45-75 cm, 15-45 cm, and 0-15 cm
depth layers, and ambient air above the soil surface.

\(\theta_w\) is the soil volumetric water content, \(\theta_a\) the
air-filled porosity, and \(\theta_T\) is the total soil porosity.

The gas diffusion coefficient \(D_{\text s}\) was calculated according
Equation~\ref{eq-Ds} as established by Millington and Quirk in
1961.\textsuperscript{\protect\hyperlink{ref-millington1961permeability}{2}}

\begin{equation}\protect\hypertarget{eq-Ds}{}{D_{\text s} = \left( \frac{\theta_w^{\frac{10}{3}} + D_{\text fw}}{H} + \theta_a^{\frac{10}{3}} \times D_{\text fa} \right) \times \theta_T^{-2}}\label{eq-Ds}\end{equation}

Here, \(H\) represents a dimensionless form of Henry's solubility
constant (\(H'\)) for N\textsubscript{2}O in water at a given
temperature. The constant \(H\) for N\textsubscript{2}O is calculated as
follows:

\begin{equation}\protect\hypertarget{eq-H}{}{H = \frac{8.5470 \times 10^5 \times \exp \frac{-2284}{\text T}}{\text R \times \text T}}\label{eq-H}\end{equation}

Here, \(\text R\) is the gas constant, and \(\text T\) is the
temperature (\(\text T = 298 \; \text K\)).

\(D_{\text{fw}}\) was calculated according to Equation~\ref{eq-Dfw} as
documented by Versteeg and Van Swaaij
(1988).\textsuperscript{\protect\hyperlink{ref-versteeg1988solubility}{3}}

\begin{equation}\protect\hypertarget{eq-Dfw}{}{D_{\text{fw}} = 5.07 \times 10^{-6} \times \exp \frac{-2371}{\text T}}\label{eq-Dfw}\end{equation}

\begin{equation}\protect\hypertarget{eq-Dfa}{}{D_{\text{fa}} = D_{\text{fa, NTP}} \times \left( \frac{\text T}{273.15} \right)^n \times \left( \frac{101'325}{\text P} \right)}\label{eq-Dfa}\end{equation}

\hypertarget{state-function-set}{%
\subsection{State function set}\label{state-function-set}}

Still to do.

\hypertarget{the-data}{%
\section{The data}\label{the-data}}

The study uses data collected from a mesocosm experiment -- i.e.~an
outdoor experiment that examines the natural environment under
controlled conditions. The experiment was set up as a randomized
complete block design, with 4 varieties and 3 replicates, using 12
non-weighted lysimeters. A non-weighted lysimeter is a device to measure
the amount of water that drains through soil, and to determine the types
and amounts of dissolved nutrients or contaminants in the water. Each
lysimeter had five sampling ports with soil moisture probes and
custom-built pore gas sample, at depths of 7.5, 30, 60, 90 and 120 cm
below soil surface.

\begin{equation}\protect\hypertarget{eq-dimension}{}{4 \times 3 \times 5 \times 161 = 9660}\label{eq-dimension}\end{equation}

Equation~\ref{eq-dimension} shows how many observations we should expect
to have. In reality, some observations are missing.

\hypertarget{references}{%
\section*{References}\label{references}}
\addcontentsline{toc}{section}{References}

\hypertarget{refs}{}
\begin{CSLReferences}{0}{0}
\leavevmode\vadjust pre{\hypertarget{ref-massman1998review}{}}%
\CSLLeftMargin{1. }%
\CSLRightInline{Massman, W. A review of the molecular diffusivities of
H2O, CO2, CH4, CO, O3, SO2, NH3, N2O, NO, and NO2 in air, O2 and N2 near
STP. \emph{Atmospheric environment} \textbf{32}, 1111--1127 (1998).}

\leavevmode\vadjust pre{\hypertarget{ref-millington1961permeability}{}}%
\CSLLeftMargin{2. }%
\CSLRightInline{Millington, R. \& Quirk, J. Permeability of porous
solids. \emph{Transactions of the Faraday Society} \textbf{57},
1200--1207 (1961).}

\leavevmode\vadjust pre{\hypertarget{ref-versteeg1988solubility}{}}%
\CSLLeftMargin{3. }%
\CSLRightInline{Versteeg, G. F. \& Van Swaaij, W. P. Solubility and
diffusivity of acid gases (carbon dioxide, nitrous oxide) in aqueous
alkanolamine solutions. \emph{Journal of Chemical \& Engineering Data}
\textbf{33}, 29--34 (1988).}

\end{CSLReferences}



\end{document}
